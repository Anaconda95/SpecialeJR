In December 2019 the Danish parliament signed a law committing Denmark to reduce greenhouse gas emissions by 70 pct. relative to 1990. Many economists believe that a uniform tax on CO2-equivalents is the most cost-effective way to reduce greenhouse gas emissions \citep{mankiw2009smart}, and the Danish Council on Climate Change have among others recommended such a carbon tax reform \citep{klimaraad2021, Kraka2020, dmoer2021}. In spite of these recommendations, parliament has so far been reluctant to pass such a reform. The Danish Minister for Climate, Energy and Utilities, Dan Jørgensen, expressed his concerns regarding carbon taxation in the Danish newspaper Berlingske: \textit{'We need to complete the green transition (...) without compromising social balance or competitiveness'} \citep{berlingske2020}. Social balance is one of 'guiding principles' of the climate law which also includes maintaining competitiveness and employment, sound public finances and securing the welfare state \citep{klimaloven}. \cite{klimaraad2021} defines social balance as being \textit{'highly dependent on the economic distribution. Social balance requires that differences in prosperity between, for example, cities and the countryside or (...) income groups do not become to great.} 

Currently, a majority in the parliament has agreed to implement uniform carbon taxes in 2030 as part of a broader reform of the Danish system of energy taxation, but have not yet agreed on the carbon tax rate or the design of the reform. Instead, a group of experts with several prominent Danish economists have been asked to design possible tax reforms taking the 'guiding principles' of the climate law into account \citep{gronskattereform2020}.

The concern that a carbon tax reform will be unpopular among voters might explain the reluctancy of parliament to implement it. \cite{klenert2018making} cites political acceptability as the biggest challenge to implementing ambitious carbon pricing schemes. Very recently, increased carbon taxation levied on gasoline and plane tickets was rejected in a Swiss national referendum. The 'no'-side argued  that the reform would mean that only the rich could afford to drive a car and buy plane tickets. The result was a surprise for many, as the Swiss experienced a 'green wave' in the 2019 parliamentary elections, and revenues from the carbon taxes were to be mostly used on tax rebates \citep{swissreferendumjune2021}. This was a clear illustration of an often repeated point in the literature on carbon taxation, namely that (perceived) regressive distributional impacts of carbon taxes decrease their political acceptability \citep{poterba1991tax, BARANZINI2000, klenert2018making,meta_carbontax}. Ensuring that a carbon tax reform is socially equitable thus might be a necessary condition of actual implementation.

This thesis examines the distributional impact of increased carbon taxation on consumption possibilities of different income groups. We analyze how price increases of consumer goods will impact consumers differently across the income distribution. The impact is quite different across income groups. For example, low income households spend a relative high share of their income on GHG-intensive foods such as meat and dairy as well as electricity and gas, but a relatively low share on transport fuels such as gasoline and diesel compared to high-income households. The impact will not only depend on the relative consumption shares but also the necessity of certain goods for different households, as well as potential exemptions to the tax and changes to other parts of the existing energy taxation. The purpose of our research is to assess the distributional impact of increased carbon taxation across the income distribution, to help qualify the debate on how to implement a carbon tax reform that is 'socially balanced'. To our knowledge, only one study, \cite{Kraka2020} has assessed the distributional impact of a carbon tax reform in a Danish context since the national climate reduction goal of 70 pct. was adopted. This thesis is a contribution to that research agenda.

In our empirical analysis we estimate demand systems for households representing quintiles of the Danish income distribution measured in disposable income. To do this, we use data from the Household Budget Survey \citep{hhbudgetsurvey}, containing consumption data of more than 1200 consumer goods and services for each quintile. We aggregate the data into 8 consumption composites: Meat and dairy, other foods, housing, energy for housing, energy for transport, transport services, other goods and other services. On these 8 consumption good composites, we estimate a linear expenditure system \citep{stone1954linear} for each quintile as well as the average household. Our preferred specification incorporates a type of habit formation, such that consumption of a given good in one year affects demand of it the following year.

We find that low income households seem to have lower price elasticities for the food composites than the rich. For the rest of the composites, price elasticities do not differ significantly across the income distribution. We also find that the income elasticity is relatively higher for low income households when it comes to energy for transport as well as transport services, indicating that these goods are to a higher degree luxury goods for those households.

We then proceed to analyze how a carbon tax reform will affect households across the income distribution. We set up an input-output model utilizing industry-level emissions data to model resulting price changes of a carbon tax reform. We couple the emissions data from 2018 with emission projections from \cite{kf21}, as emissions from especially the electricity sector are expected to fall significantly towards 2030. To evaluate the distributional impact, we calculate equivalent variations for each income quintile using a partial demand model and our estimates for the linear expenditure system. We find that carbon taxes are distributionally neutral when compared to a proxy of lifetime income, but regressive compared to annual disposable income. The neutral distribution effect is a result of off-setting effects: the poor are hit relatively harder through price increases of foods, electricity and heating, while the rich are hit relatively harder through higher transport fuel prices. 

To assess general equilibrium effects on the price levels, as well as assessing potential wage effects, we run experiments with the dynamic CGE model GreenREFORM, which is under development under the Danish Ministry of Finance. We then apply these price and wage effects to our partial demand model. We find that when including general equilibrium effects, the aggregate price changes are actually negative in the short run, and only slightly positive after 2030. This is especially due to a reduction in prices of firms' inputs, such as labor, services including rental of buildings, and land for the agricultural sectors. This leads to a very small tax burden burden for households based on the consumption of goods. The burden is neutral as share of total expenditure but slightly progressive in the short run, and slightly regressive after 2030, as share of disposable income. The drop in the price of labor, on the other hand, implies a reduction in the households' incomes. How this wage drop is distributed across quintiles of the income distribution is difficult to assess without modeling 5 heterogeneous consumers within the CGE model.  

Based on the two analyses we conclude that a carbon tax of between 1250 DKK on top of existing taxes and 1500 DKK uniformly will affect consumer prices in a way that is approximately neutral as share of total expenditure, but not equally distributed as share of disposable income. When the tax burden is fully passed onto consumers (as in the input-output model), the price increases makes the households consumption bundle 1 percent more expensive for the top quitile, and 2 percent more expensive for the bottom quintile, measured as share of disposable income. Considering general equilibrium effects, the households' consumption bundle becomes just 0.001 percent more expensive for the top quintile, and 0.0025 percent more expensive for the bottom quintile (after 2030), measured as share of disposable income. The tax is instead pushing down wages, which drops with around 2 percent on average.

Our findings from the input-output analysis can be used as an 'upper bound' for the households' tax burden due to increased prices from a domestic carbon tax in Denmark. When considering lowered demand, subsitution effects and especially foreign competition, the GreenREFORM analysis shows that price increases are small (and positive in the very short run), while wage effects are larger. If Denmark as the only country imposes a domestic carbon tax, it should be a concern to policy makers to assess the distributional effects of wage drops due to a carbon tax. However, if other countries that share markets with Denmark also impose carbon taxes, this will reduce the effect of foreign competition, which will reduce wage effects and increase price effects. 



 



