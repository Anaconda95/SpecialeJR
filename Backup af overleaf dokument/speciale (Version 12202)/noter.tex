Taget ud fra introduction

There are other factors that can influence distribution of a carbon tax that we have not looked into. We have not considered the effect on asset prices, which is likely to have a progressive outcome. Further, the distribution of the tax burden between urban and rural households is likely to be skewed due to the large price increase of energy for transport. Nor have we looked at unemployment effects or revenue use.

Another argument for considering non-unitary income elasticities is Pigou's law. It states that under certain assumptions on preferences,\footnote{Preferences should be characterized by additively separable utility and constant marginal utility of income, and goods should have a neglible budget share \citep{snow2015pigouslaw}.} income and own price elasticities are proportional \citep{snow2015pigouslaw}. This has important consequences for evaluating the distributional impact of carbon taxes. If income elasticities of demand vary across the income distribution, Pigou's law implies that own price elasticities vary proportionally. 

