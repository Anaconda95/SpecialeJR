In this section we review the literature on the distributional impact of carbon taxation in both Denmark and abroad. We focus on both methodology and evidence. We further discuss the demand systems used in such analyses, how they are estimated and their strengths and weaknesses with regards to capturing relevant demand effects.

\subsection{The distributional impact of carbon taxation}

The literature on the interplay between carbon taxation and inequality has been growing in the recent year. A simple search on the Web of Science\footnote{See \hyperlink{https://clarivate.com/webofsciencegroup/solutions/web-of-science/}{https://clarivate.com/webofsciencegroup/solutions/web-of-science/}} on the terms "carbon tax" and "inequality" yielded 3 articles in 2001-2010 and 37 in 2011-2020. This makes sense as concerns over the distributional impact of carbon taxation has become more prevalent \cite{klenert2018making}, as well as the overall concern for climate change worldwide

\subsubsection{Measuring the distributional impact}\label{sec:measuringdistribution}
The distributional impact of carbon taxation may be measured across several dimensions: between income groups, different household types, between rural and urban households or between generations. The present study will, as the majority of exisiting studies in that regard \citep{BARANZINI2000,meta_carbontax}, focus on the distributional impact across the income distribution. 

We will, as \cite{BARANZINI2000} and \cite{Wier2005}, call a carbon tax reform \textit{regressive} if the tax burden falls disproportionately on low-income households, that is, that the tax burden is higher measured as a share of their annual total expenditure\footnote{We use expendiure as a proxy for lifetime income, see section \ref{sec:incomeorexpenditure}.} or disposable income .

As \cite{BARANZINI2000} note, the distributional impacts of a carbon tax are 'quite complicated' and depend on at least the four following factors: 
\begin{enumerate}
    \item \textit{Households' expenditure structure.} The direct purchases of fossil fuels, such as heating oil, natural gas and transport fuels, as well as the purchase of all other goods, which to different degrees have emitted greenhouse gasses in the production process. Households' expenditure structure differ quite significantly across the income distribution, as we show in section \ref{sec:data}.
    \item \textit{Tax incidence.} Will increased carbon taxation lead to higher consumer prices or will fossil fuel producers bear the burden as lower profits? 
    \item \textit{The distribution of benefits from improved environment quality.} The environmental benefits from increased carbon taxation may not be distributed equally among the population. For example, increased carbon taxation may have a side effect of improving local air quality of cities, thus affecting urban but not rural households. 
    \item \textit{Revenue use.} The use of fiscal revenues can ex-post improve the progressivity of a carbon tax reform, for example through lump-sum transfers or decreased labor taxation.
\end{enumerate}
We add a fifth factor, namely income effects:
\begin{enumerate}[resume]
    \item \textit{Income effects.} Carbon taxation will lower demand for goods, which will most likely reduce wages and/or increase unemployment. Further, if firms' profits are reduced, this can lead to lower asset prices, which will disproportionately affect wealthy households. 
\end{enumerate}
In this thesis our main focus is to analyze households' expenditure structure across the income distribution and how different income groups may react differently to price changes. We also look into tax incidence and wage effects by applying a general equilibrium model. However, the benefits of improved environmental quality and revenue use is beyond the scope of this paper. 


\subsubsection{Evidence from Denmark} 
 
In Denmark, a few studies have evaluated the distributional impact of carbon taxation in recent years.

\cite{Wier2005} measure the regressivity of the existing carbon tax in Denmark using data from 1996. They use an input-output model combined with data from the household budget survey to calculate the tax incidence on households across the income distribution, as well as between urban and rural households. They find that existing carbon taxes were indeed regressive measured as a share of disposable income, but approximately distributionally neutral as a share of total expenditure. 

\cite{dmoer2009} set up a static CGE model with 130 industries and 11 consumers to evaluate the impact of increased carbon taxation, among other environmentally related taxes. They find that carbon taxation is progressive measured relative to consumption, mostly because richer households spend a larger share of their total expenditure on carbon intensive transport goods. They also argue that the distributional impact ultimately is very dependent on the revenue redistribution. If the revenue is spent on equally sized (in DKKs) tax deductions, poorer households will be better off than before. 

\cite{Kraka2020} use the general equilibrium model REFORM to calculate price and quantity changes on consumption good composites resulting from increased carbon taxation. Using data from the Danish household budget survey, they calculate the distributional impact of a carbon tax reform under different revenue recycling schemes. In their preferred recycling scheme, where other energy taxes decrease and carbon intensive industries are given carbon tax exemptions, they find that the effect on the income distribution is neutral. In another recycling scheme, where the carbon tax revenue is paid out as lump sum transfers, they find that a carbon tax reform is progressive. Without recycling of the revenue, they do find that their carbon tax reform is regressive relative to both disposable income and total expenditure. One key assumption of the analysis is that all income deciles have the same price and income elasticity for all goods. 

\cite{sune2020} finds, like \cite{Wier2005} that the impact carbon tax increase of 1,000 DKK is neutral across the income distribution relative to total expenditure, but regressive measured relative to disposable income. The method he uses is inspired by \cite{Wier2005}.
\subsubsection{Evidence from abroad}
The literature on the distributional impact of carbon taxation in various forms is quite large. They range from economy-wide CGE model assesments to more narrow analyses of the transport sector or household electricity consumption. 

\cite{meta_carbontax} is a metaanalysis of 53 empirical studies of the distributional impact of carbon taxation. Their sample consists of studies considering economy-wide effects of carbon taxation, such as CGE models or studies like \cite{Wier2005}, and studies of the transport sector. One of their main results is that increased carbon pricing in the transport sector increases the likelihood of a progressive outcome across countries. They also find that studies that used lifetime income proxies as opposed to annual household incomes generally find more progressive results, and that carbon pricing is more often associated with progressive outcomes in countries with lower income levels. 

Overall, the evidence suggests that carbon taxation in developed countries is not regressive, at least not measured relative to lifetime income. Measured relative to annual disposable income, it is more likely to be regressive.

\subsection{A review of consumer demand analysis}
A popular way of estimating the impact of a specific policy is to estimate a demand system and calculate welfare effects \citep{shojaeddiniconsumer}. In this thesis, we continue the tradition. In this section we describe different demand systems and their advantages and disadvantages. We conclude that an LES demand system is best suited for measuring the impact of a carbon tax, given our dataset and our perceived importance of non-unitary income elasticities.

The economic impacts of environmental regulation such as carbon taxation is often assessed as the difference in a welfare variable between a baseline and a policy case. The specification of the consumer demand system is an  important determinant of many such variables. It plays both a role in the baseline, as incomes and prices can change over time which has consequences for the expenditure structure, and of course in the policy case determining final good demands. Consumer demand systems also is a determinant of the distribution of the abatement costs between consumers and firms \citep{shojaeddiniconsumer}. In this section, we consider the advantages and disadvantages of different consumer demand systems when estimating the distributional impact of carbon taxation.

Estimation strategies are varied and plentiful. Often, focus is on estimating consumer demand for a specific good or sector with a flexible functional form suchs as the QES, AIDS or QUAIDS \citep{pollak1992demand,shojaeddiniconsumer}. While these flexible functional forms are well suited for representing consumer demand in a very detailed way, they are relatively difficult to interpret and to implement in CGE models. 

In CGE models, a typical approach is to calibrate a relatively simple demand system, such as the CES, the LES or the CDE to existing parameter estimates \citep{shojaeddiniconsumer}. This is, however, not uncontroversial. For example, it does not consider the confidence intervals of estimation, and available elasticity estimates are not regularly updated \citep{hertel2007confident}. It is also important that the underlying behavioral assumptions in the estimation framework are consistent with those in the model.

In the reviewed literature on the distributional impact of carbon taxation, not much attention has been paid to the choice and estimation of demand system for consumers. \cite{dmoer2009} assume an identical CES utility function for each consumer, and \cite{Kraka2020} assumes that each consumer acts as the representative consumer in their CGE-model, which also has CES-utility.

\subsubsection{The CES utility function and its limitations}
A CES utility function is homothetic of nature and has unitary income elasticity \citep{annabi2006functional}. The assumption of unitary income elasticity is potentially quite restrictive in the analysis of the distributional impact of climate policies. For example, Engel's law dictates that the income elasticity of food is less than one, a result that has been remarkedly consistent across time and space \citep{kaus2013beyond}. This implies that if prices generally rise as a result of a carbon tax reform or real wages decrease and real income falls, low income households will lower their food consumption to a lower extent than high income households. Further, if the \textit{subsistence} consumption of food is closer to the actual consumption of low income households, they will have a smaller own price elasticity, and carry a relatively higher tax burden if carbon taxation increases consumer prices of food.

\cite{KlenertMattauch2016} and \cite{klenert2018} analyze the role of subsistence consumption in the regressiveness of carbon taxation. In the latter study, they set up a CGE-model where households have non-homothetic preferences and need a minimal level of 'polluting consumption'. These preferences play an important role in their model. The regressive effect of carbon taxation is removed when the subsistence level is set to zero. Thus, at least theoretically, subsistence levels of consumption of certain goods could mean a lot to the distributional impact of carbon taxation. 

\citet[p. 24]{pollak1992demand} argue that a CES or other homothetic demand systems are too simple to even serve as a first approximation of empirical demand analysis. Expenditure proportionality is generated from homothetic utility functions. Expenditure proportionality implies that the demand for all goods is proportional to expenditure. The LES is a non-homothetic demand system, and therefore the demand for goods is not necessarily proportional to income. This further implies that the income elasticity may differ from 1. 

\cite{pollak1992demand} also argue that the LES is inferior to a wide range of alternative specification, including their own quadratic expenditure system QES. However, estimating a LES gives a more straightforward interpretation of parameters and is simpler to estimate.
Thus, we choose to estimate an LES demand system due to its simple interpretation and its non-homothetic properties. 

\cite{annabi2006functional} gives an overview on the choice of functional forms of consumer demand systems. They argue that choosing a linear expenditure system (LES) is a good choice to take account of that fact that budget shares do vary with income. Another advantage of the LES is that it, in its most basic form, has relatively few parameters, which is important given that we will estimate it on annual time-series data with only 26 observations. 



\subsubsection{Estimation of the linear expenditure system}
\cite{pollak1969estimation} is one of the first papers to estimate the LES. They use time series US consumption data. They also introduce the functional forms of habit formation that we use in this paper, and find that estimation of the parameters is very sensitive to both the functional form of habit formation and the estimation procedure. 

The literature on estimating complete LES consumer demand system is quite small. In recent years, we have only found two studies estimating the LES. \cite{jussila2012estimation} estimate the LES on Finnish data. They find that price and income elasticities generally do no vary much across income deciles. \cite{gharibnavaz2018estimating} estimate a linear expenditure system on Australian household budget survey data for five household types differentiated by income. They estimate on cross-sectional data usinge the seemingly unrelated equations estimation procedure (SURE). They find that the subsistence consumption parameters are relatively higher for lower income quintiles. They also find that certain manufacturing goods, insurance and finance services, telecommunication, among others, to a higher degree are luxury goods for the lower income quintiles. Both of these studies employ cross-sectional data, which means that their methods are not directly applicable to our data set, but we will compare our results to theirs.

\subsection{Differences in expenditure structure and elasticities}
As we noted in section \ref{sec:measuringdistribution}, an important determinant of the distributional impact of carbon taxation is the expenditure structure of different households. 

Danish lower income households generally spend a higher fraction of their income on carbon intensive goods such as food, electricity and heating.\footnote{High income households spend a larger fraction of their income on transport fuels, another relatively carbon intensive good. See also section \ref{sec:data}.} Given that these goods can be considered necessary to some extent, and if lower income households are closer to their subsistence consumption level, price and income elasticities should differ across the income distribution. In that case, analyses where the elasticities are assumed equal across the income distribution could wrongfully estimate the distributional impact of carbon taxes.

Whether elasticities actually do differ is an empirical question that we will try to answer in this paper. \cite{harold2017incomeelasticityenergy} argue that a general assumption in the literature is that the effect on demand of a change in average income is the main concern, and thus the elasticity at the mean income is of most interest. However, as they argue, when the distributional consequences of policy is the concern, it is important to take account of potential variation in the elasticity across the income distribution. They estimate the income elasticity of demand of household energy to vary quite a bit across the income distribution. More specifically, low income households tend to have a higher income elasticity than high income households.

Another example of the heterogeneity of demand could be the case of transport fuels. Lower income households may be already driving as little as possible because of their budget constraint and are thus unable to reduce their consumption of fuel further. Higher income households may more easily reduce transport fuel for leisure trips or switch to more fuel efficient vehicles. \cite{wadud2010gasoline} show that different income groups have differing behavioral responses to changes in fuel price, estimated on American household budget survey data. They find that the price elasticity of gasoline is highest for the lowest quintile and falls across the income distribution. They also find that the income elasticity decreases as income increases, and suggest that it could be due to demand satiation at a higher income level. \cite{schulte2017price} estimate on German data that the expenditure elasticity of electricity rises across the income distribution, while the price elasticity is much lower for the lowest quartile than for the highest. 

These results indicate that price and income elasticities can vary across the income distribution, and that this may be relevant in assessing the distributional impact of carbon taxation. 